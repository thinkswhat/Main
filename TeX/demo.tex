\documentclass{Krep}
\title{Krep.cls模板使用示例}
\author{KONG}
\date{\today}
\addbibresource{references.bib}
\Lhead{左页眉示例}
\Cfoot{\thepage}

\begin{document}
\maketitle
\begin{abstract}
	在这个示例教程中,我们将演示如何使用Krep.cls模板来创建一个简单的中文文档。这个文档包括数学公式、表格、图片、列表、代码和参考文献等内容。
\end{abstract}
\section{数学公式示例}

这是一个简单的数学公式示例:

\begin{equation}
	E = mc^2
\end{equation}
\section{表格示例}

这是一个简单的表格示例:

\begin{table}[h]
	\centering
	\begin{tabular}{|c|c|c|}
		\hline
		项目 & 数量 & 单价 \\
		\hline
		A    & 5   & 20   \\
		\hline
		B    & 3   & 25   \\
		\hline
	\end{tabular}
	\caption{一个简单的表格}
\end{table}
\section{图片示例}

这是一个简单的图片示例:

\begin{figure}[h]
	\centering
	\includegraphics[width=0.5\textwidth]{example-image}
	\caption{一个简单的图片}
\end{figure}
\section{列表示例}

这是一个简单的列表示例:

\begin{itemize}
	\item 列表项 1
	\item 列表项 2
	\item 列表项 3
\end{itemize}
\section{代码示例}

这是一个简单的Python代码示例:

\begin{minted}[linenos=true, frame=single]{python}
    def hello_world():
    print("Hello, World!")
    
    hello_world()
\end{minted}

\section{伪代码示例}
这是一个简单的算法伪代码示例:

\begin{algorithm}[H]
	\caption{计算斐波那契数列的第n项}
	\KwIn{正整数$n$}
	\KwOut{第$n$项斐波那契数列的值}
	\SetKwProg{Fn}{Function}{\string:}{end}
	\Fn{Fibonacci($n$)}{
		\If{$n=1$ or $n=2$}{
			\Return 1;
		}
		$F_1\leftarrow 1$; $F_2\leftarrow 1$;
		\For{$i\leftarrow 3$ \KwTo $n$}{
			$F_i\leftarrow F_{i-1} + F_{i-2}$;
		}
		\Return $F_n$;
	}
\end{algorithm}
\section{参考文献示例}

在这个示例中,我们引用了一篇文章\cite{ref1}作为示例。

\printbibliography
\end{document}